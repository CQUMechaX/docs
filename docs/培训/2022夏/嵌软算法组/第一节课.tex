% \title{工欲善其事 必先利其器}
%\lstset{
%  frame=l,
%  basicstyle=\ttfamily,
%  numbers=left,
%  numberstyle=\tiny\color{gray},
%  keywordstyle=\color{blue},
%  commentstyle=\color{green}\ttfamily,
%  stringstyle=\color{mauve}
  %texcl=true 
%}

\section{我是谁}

\begin{frame}{我是谁}
    小学:FLL 2012 关爱老人。
    
    初中:完全没看懂 \url{https://modthesims.info/showthread.php?t=536436}在干什么。
    
    高中:搞了会信竞。
    
    大学:看周围的人拿奖。
\end{frame}

\begin{frame}{Robomaster 机甲大师高校赛}
    \begin{columns}
    \begin{column}{0.4\textwidth}
    \begin{itemize}
        \item \emoji{zany-face} Battlebots
        
        \emoji{zany-face} LEGO Master
        
        \item \emoji{smiling-face-with-halo} FIRST Robotics Competition
        
        \emoji{smiling-face-with-halo} RoboCON
        
        \emoji{smiling-face-with-halo} CADC 中国飞行器设计挑战大赛
    \end{itemize}
    \end{column}
    \begin{column}{0.7\textwidth}
    \begin{figure}
        \centering
        \includegraphics[width=.6\textwidth]{docs/培训/2022夏/images/class1/RMUC.png}
        %\includegraphics[]{ProjectRoot/docs/培训/2022夏/images/class1/doc3.jpg}
        \caption{超级对抗赛场地}
        \label{fig:my_label}
    \end{figure}
    %\includegraphics[width=\textwidth]{images/default}
    \end{column}
    \end{columns}
\end{frame}

\section{课前知识回顾}

\begin{frame}{Modern C++} %fragile
    以 \href{https://github.com/ros/ros_tutorials/blob/humble/turtlesim/src/turtle_frame.cpp}{turtlesim} 为例。
    
    见 example.cpp 等。
\end{frame}

\begin{frame}{Python Trick} %fragile
    \begin{itemize}
        \item rhel 链接
        \item tab with exception
        \item annotation
        \item generator
        \item -> map-reduce
        \item decorate
        \item scope and underscore
    \end{itemize}
    
    见 example.py 等。
\end{frame}

\section{标准与实现}

\subsection{实例}

\begin{frame}{工程实例}
    \begin{columns}
    \begin{column}{.5\textwidth}
    \centering
    \includegraphics[width=.6\textwidth]{docs/培训/2022夏/images/class1/std_xh.png}
    
    XH 2.54
    \end{column}
    \begin{column}{.5\textwidth}
    \centering
    \includegraphics[width=.8\textwidth]{docs/培训/2022夏/images/class1/std_jlc.png}
    
    \href{https://shimo.im/docs/0765b0f4a4704ecb/read}{立创EDA封装库命名参考规范}
    \end{column}
    \end{columns}
\end{frame}

\begin{frame}{编程语言实例}
    \begin{itemize}
        \item \href{https://peps.python.org/pep-0328/}{PEP328: Imports: Multi-Line and Absolute/Relative}
        
        \href{https://peps.python.org/pep-0517/}{PEP517: A build-system independent format for source trees}
        
        \item isocpp v.s. cppreferences
        
        \url{http://open-std.org/jtc1/sc22/wg21/docs/papers/2022/n4910.pdf}
        
        \url{https://isocpp.org/wiki/faq/operator-overloading}
        
        \item codelint
        \begin{itemize}
            \item 可以解决什么和不能解决什么。
            \item \url{http://isocpp.github.io/CppCoreGuidelines/CppCoreGuidelines}
            \item \url{https://github.com/ryanmcdermott/clean-code-javascript}
        \end{itemize}
    \end{itemize}
\end{frame}

\begin{frame}{标准机构}
    \begin{itemize}
        \item The International Organization for Standardization
        \item Internet Engineering Task Force
        \item The IEEE Standards Association
    \end{itemize}
\end{frame}

\subsection{为什么需要标准}

\begin{frame}{自由讨论}
    
\end{frame}

\begin{frame}{工程问题下的标准}
    \begin{itemize}
        \item 背景:黑箱
        \item 需求:快速迭代
        \item 目标:高可靠性
    \end{itemize}
\end{frame}

\begin{frame}{实例:serial}
    \begin{itemize}
        \item \url{https://github.com/yunwaikongshan/RM2020-Horizon-InfantryVisionDetector/blob/master/include/serial.h}
        
        \url{https://github.com/xinyang-go/SJTU-RM-CV-2019/blob/master/others/include/serial.h}
        
        \item \url{https://github.com/chenjunnn/rm_vision/blob/main/rm_serial_driver/include/rm_serial_driver/rm_serial_driver.hpp}
        
        \url{https://github.com/ros-drivers/transport_drivers/blob/main/serial_driver/include/serial_driver/serial_port.hpp}
        
        \item boost::asio
    \end{itemize}
\end{frame}

\begin{frame}{实例:apt dependency}
    \url{https://github.com/cqumechax/docs/blob/5e20/docs/培训/2022夏/images/class1/galactic_ros_core.svg}
\end{frame}

\subsection{如何设计标准}

\begin{frame}{标准用来解决问题}
    向前兼容性
    
    问题针对性
    
    设计科学性
\end{frame}

\section{其他杂项}

\subsection{版本}

\begin{frame}{Semantic Versioning}
    \url{https://semver.org/lang/zh-CN/}
\end{frame}

\begin{frame}{软件包管理器}
    
\end{frame}

\subsection{文档}

\begin{frame}{标点空格符号 / 默认值缺省值}
    \centering
    \includegraphics[width=.3\textwidth]{docs/培训/2022夏/images/class1/doc1.jpg}
\end{frame}

\begin{frame}{避免软件偷懒 / 学会收拾残局}
    \centering
    \includegraphics[width=.4\textwidth]{docs/培训/2022夏/images/class1/doc2.jpg}
\end{frame}

\begin{frame}{提取关键信息 / 判断错误根源}
    
\end{frame}


\section{课后任务}

\begin{frame}{基础任务}
    使用钉钉作业功能提交。注意文字、代码、视频内容的提交规范。
    
    时间要求:最好在下次上课前提交。在存档前提交,都尽可能批改。
    
    \begin{itemize}
        \item 选择一个问题,查询标准和(动手操作)相应实现来解答问题。
        \item 阅读课前/后资料,并结合新的例子,指出标准和实现文档一般会在哪些地方出现差异
    \end{itemize}
    
\end{frame}

\begin{frame}{进阶任务}
    鼓励合作:选择课题与人员,实时更新进度,善于提问交流,形成成果分享。
    
    (初定)暑期中展示交流一次,开学后展示交流一次。各组展示与提问 5 - 10 分钟。
    
    \begin{itemize}
        \item 为两种或多种常见的编程语言,列表枚举其写法/功能异同,并寻找对应功能的有趣实现(如功能包)。
    \end{itemize}
\end{frame}